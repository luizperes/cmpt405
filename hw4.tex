\documentclass{article}
\PassOptionsToPackage{hyphens}{url}
\usepackage[hidelinks]{hyperref}
\usepackage{amsmath}
\usepackage{amsthm}
\usepackage{amssymb}
\usepackage{pgfplots}
\usepackage{algpseudocode}
\newcommand{\QED}{\hfill {\qed}}
\newcommand\tab[1][1cm]{\hspace*{#1}}
\usepackage{mathtools}
\DeclarePairedDelimiter\ceil{\lceil}{\rceil}
\DeclarePairedDelimiter\floor{\lfloor}{\rfloor}
\graphicspath{ {./imgs/} }

\title{\#4 Assignment - CMPT 405}
\author{Luiz Fernando Peres de Oliveira - 301288301 - lperesde@sfu.ca}

\begin{document}

\maketitle
\textbf{\#1}
\\
Let $w_{ij}$ be the weight of every $(i, j) \in E$ and $x_{ij}$ be variables such that $x_{ij} = 1$ if the shortest path contains $i \rightarrow j$ and $x_{ij} = 0$, otherwise. The shortest path from a source $s \in V$ to a target $t \in V$ in a weighted graph $G=(V, E, \textit{w})$ can be found by minimizing the summation of $w_{ij}x_{ij}$ for every $(i, j)$. See below:
\begin{gather*}
x_{ij} =
\begin{cases}
1 \tab\tab\text{ if the shortest path contains } i \rightarrow j \\
0 \tab\tab\text{ otherwise}\\
\end{cases}
\end{gather*}
\\
By the principle of network flow, we have that for each single node $i$, the amount of a flow $f_i$ is equal the difference between the amount of outgoing flow from $i$ and the amount of incoming flow to $i$:
$$
f_i = \sum_{j}x_{ij} - \sum_{k} x_{ki}
$$
\\
As we are looking for the shortest path from $s$ to $t$, we know that our network will "travel" from the source to the target, cancelling any flow $f_u$ for single vertices $u$ between $s$ and $t$ in our network, where $u \neq s$ and $u \neq t$. Because there is no incoming flow in $s$, $f_s = 1$. Likewise, because there is no outgoing flow in $t$, $f_t = -1$. Thus:
\begin{gather*}
f_i =
\begin{cases}
1 \text{ } \tab\tab\text{ if } i = s\\
-1 \tab\tab\text{if } i = t\\
0 \tab\tab\text{ otherwise}
\end{cases}
\end{gather*}
\\
Assuming that $x_{ij} \geq 0$, the linear program for the shortest problem is:
$$
min \sum_{(i, j) \in E} w_{ij}x_{ij}
$$
\\
The resulting dual will have one variable $y_u$ for each vertex $u$ in the  graph. The values of $y$ have the constraint that $y_j - y_i \leq w_{ij}$ and the objective function is the maximization of $y_s - y_t$:
\\
$$
max \text{ } y_s - y_t
$$
$$
y_j - y_i \leq w_{ij} \text{ , } \forall (i,j) \in E
$$
\textbf{Dual Encoding:}
The dual can be interpreted as the encoding of Bellman-Ford, because when BF terminates, it has computed for each vertex $j$ a value $y_j$, such that for each edge $(i,j) \in E$, we have the same constraints as the dual: $y_j \leq y_i + w_{ij}$. The objective function is also the maximization of $y_s - y_t$.
\\
\\
\textbf{\#2}
\\
\\
In a similar way of question \#1, let $w_e$ be the weight of every $e \in E$ and $x_e$ be $0-1$ variables such that $x_e = 1$ if the edge $e$ is in the matching and $x_e = 0$, otherwise. 
\begin{gather*}
x_e =
\begin{cases}
1 \tab\tab\text{ inclusion of edge } e \text{ in the matching} \\
0 \tab\tab\text{ otherwise}\\
\end{cases}
\end{gather*}
We need to choose at each step an augmenting path $p$ that produces the largest possible increase in weight so that the matching obtained by flipping the edges has maximum weight.
\\
\\
The objective function maximizes the weight of all edges $e$ in the matching and, because we are augmenting paths in our matchings, we use constraints to limit one edge per vertex so that the path is created in the form $x_e \leq 1$, for all vertices $u$, such that $e=(u, v)$. The linear program is then:
$$
max \text{ } \sum_e w_e x_e
$$
$$
\sum_{e=(u,v)} x_e \leq 1 \text{ , } \forall u \in V
$$
\\
\\
\textbf{\#3}
\\
To prove that the constraint matrix in previous example is totally unimodular for bipartite graphs, we must prove that every entry is equal to $1, 0$ or $-1$.
\\
\\
Let $x_{ij}$ be an alias for $x_e$ in the previous example, such that $e=(i,j)$. Each row $i$ in $x$ represent a vertex $v_i$ and each column $j$ represents an edge $e=(i, j)$. $x_{ij} = 1$ only if our previous algorithm included $(i, j)$ in the matching.
\\
\\
\begin{proof}
Consider an arbitrary square submatrix $x'$ of $x$. The goal here is to show that the determinant of $x'$ is in $\{ 1, 0, -1\}$.
\\
\\
\textbf{Case 1:} $x'$ has a column with only 0. Then the determinant of $x' = 0$.
\\
\textbf{Case 2:} $x'$ has a column with only 1. By induction, $x''$ has determinant equal $1, 0$ or $-1$ and so does $x'$.
\[
x' =
\begin{bmatrix}
    1       & ... \\
    0      & x'' \\
\end{bmatrix}
\]
\textbf{Case 3:} Each column of x' has exactly two 1. Because we are dealing with a bipartite graph, we have two distinct sets $x^{[+1]}, x^{[-1]}$, $x^{[+1]} \cap x^{[-1]} = \emptyset$, such that each edge $e=(i,j) \in x$ will have $i$ in one of this sets and $j$ in the other set (otherwise it would not be bipartite). Because the rows are linearly dependent, by multiplying $+1$ in the rows in $x^{[+1]}$ and $-1$ on the columns in $x^{[-1]}$, the determinant of $x' = 0$.
\[
x' =
\begin{bmatrix}
    x^{[+1]}\\
    x^{[-1]}\\
\end{bmatrix}
\]
\end{proof}
\textbf{Counterexample for non-bipartite graphs:}
\\
In spite of working for bipartite graphs, the same does not hold for non-bipartite graphs. It can be easily proved by a simple counterexample (use the same linear program as described on \#2): $G=(V,E)$, $V=\{a, b, c\}$, $E=\{(a,b) = 1/3, (b, c) = 1/3, (a,c) = 1/3\}$.
\\
\\
\textbf{\#4}
\\
In this problem, we are trying to find the minimum value $max(x \cdot a, (1-x) \cdot b)$, for two vectors $a$ and $b$ of length $n$, $x \in \{0, 1\}^n$. The algorithm to solve such problem is as follows:
\begin{gather*}
x_i =
\begin{cases}
1 \tab\tab\text{ if we choose element $i$ in vector a}\\
0 \tab\tab\text{ if we choose element $i$ in vector b}\\
\end{cases}
\end{gather*}
$$
\textit{minimize } max(x \cdot a, (1-x) \cdot b)
$$
In order to find the minimum $max(x \cdot a, (1-x) \cdot b)$ our algorithm will try every single possibility exhaustively. For example, for $n = 3$, our algorithm would try all eight possibilities: $000, 001, 010, 011, 100, 101, 110$ and $111$. Because we have two choices at a time ($0-1$ problem), it will take $O(2^n n)$ time to be solved, meaning that is not likely to have an algorithm that will solve the problem in polynomial time as the greedy approach (always taking the smallest number at a time) will fail for some input. Take $a=[3, 4, 10]$ and $b=[5,5,9]$, for example. The DP approach can be done by modifying the $0-1$ knapsack problem (which is not polynomial as well), with utility equals weight, where we take an item from $b$ if $0$, and from $a$ if $1$. Let $M$ be the problem $max(x \cdot a, (1-x) \cdot b)$
\\
\\
\begin{proof}
\textbf{Reducing knapsack to our problem:} It is known that the knapsack problem is NP-complete (and was shown in class how to reduce it to the partition problem, by using utility equals weight and then comparing the sets). The idea of this problem is to show a reduction in polynomial time of the knapsack problem to our algorithm. Let $K$ be an alias for Knapsack problem. 
\\
To show that $K \leq_{p} M$, we only need to make a small change on the abstraction of the problem: on the knapsack problem we have the option of either $(i)$ take element $k$ or $(ii)$ not take element $k$ in a $k_{th}$ iteration. In our reduction, we keep the binary choice by using $(i)$ whenever I choose a $a_k$ element of vector $a$ and $(ii)$ whenever I choose a $b_k$ element of vector $b$ in a $k_{th}$ iteration.
\\
\\
\textbf{Reduction in polynomial time:} The reduction can be done in polynomial time, and, in fact, it does not change the complexity of the problem, because we only changed the way of interpreting the problem.
\\
\\
\textbf{$K$ has a solution iff $M$ has a solution}:
\\
Consider $W$ being the total capacity of weight of a knapsack, then:
\\
$\Longrightarrow $ If there exists a set of numbers in $K$ that sum up to less or equal than $W$, then there exists a set of some of numbers using vectors $a, b$ such that they sum up to less or equal than $W$.
\\
$\Longleftarrow $ Let’s say that there exists some numbers in $a, b$ that sum up to less or equal than $W$, then, these numbers in $K$ would have chosen some numbers in either $(i)$ or $(ii)$, that sum up to equal or less than $W$.
\\
\\
\textbf{NP-hard problem:} Because our problem $M$ was proved to be \textbf{NP-complete}, we know that its optimization version (minimum of $max(x \cdot a, (1-x) \cdot b)$) is NP-hard.
\end{proof}
\textbf{\#5}
\\
\end{document}