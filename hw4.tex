\documentclass{article}
\PassOptionsToPackage{hyphens}{url}
\usepackage[hidelinks]{hyperref}
\usepackage{amsmath}
\usepackage{amsthm}
\usepackage{amssymb}
\usepackage{pgfplots}
\usepackage{algpseudocode}
\newcommand{\QED}{\hfill {\qed}}
\newcommand\tab[1][1cm]{\hspace*{#1}}
\usepackage{mathtools}
\DeclarePairedDelimiter\ceil{\lceil}{\rceil}
\DeclarePairedDelimiter\floor{\lfloor}{\rfloor}
\graphicspath{ {./imgs/} }

\title{\#4 Assignment - CMPT 405}
\author{Luiz Fernando Peres de Oliveira - 301288301 - lperesde@sfu.ca}

\begin{document}

\maketitle
\textbf{\#1}
\\
Let $w_{ij}$ be the weight of every $(i, j) \in E$ and $x_{ij}$ be variables such that $x_{ij} = 1$ if the shortest path contains $i \rightarrow j$ and $x_{ij} = 0$, otherwise. The shortest path from a source $s \in V$ to a target $t \in V$ in a weighted graph $G=(V, E, \textit{w})$ can be found by minimizing the summation of $w_{ij}x_{ij}$ for every $(i, j)$. See below:
\begin{gather*}
x_{ij} =
\begin{cases}
1 \tab\tab\text{ if the shortest path contains } i \rightarrow j \\
0 \tab\tab\text{ otherwise}\\
\end{cases}
\end{gather*}
\\
By the principle of amount of flow network, we have that for each single node $i$, the amount of a flow $f_i$ is equal the difference between the amount of outgoing flow from $i$ and the amount of incoming flow to $i$:
$$
f_i = \sum_{j}x_{ij} - \sum_{k} x_{ki}
$$
\\
As we are looking for the shortest path from $s$ to $t$, we know that our network will "travel" from the source to the target, cancelling any flow $f_u$ for single vertices $u$ between $s$ and $t$ in our network, where $u \neq s$ and $u \neq t$. Because there is no incoming flow in $s$, $f_s = 1$. Likewise, because there are no outgoing flow in $t$, $f_t = -1$. Thus:
\begin{gather*}
f_i =
\begin{cases}
1 \text{ } \tab\tab\text{ if } i = s\\
-1 \tab\tab\text{if } i = t\\
0 \tab\tab\text{ otherwise}
\end{cases}
\end{gather*}
\\
Assuming that $x_{ij} \geq 0$, the linear program for the shortest problem is:
$$
min \sum_{(i, j) \in E} w_{ij}x_{ij}
$$
\\
The resulting dual will have one variable $y_u$ for each vertex $u$ in the  graph. The values of $y$ have the constraint that $y_j - y_i \leq w_{ij}$ and the objective function is the maximization of $y_s - y_t$:
\\
$$
max \text{ } y_s - y_t
$$
$$
y_j - y_i \leq w_{ij} \text{ , } \forall (i,j) \in E
$$
\textbf{Dual Encoding:}
The dual can be interpreted as the encoding of Bellman-Ford, because when BF terminates, it has computed for each vertex $j$ a value $y_j$, such that for each edge $(i,j) \in E$, we have the same constraints as the dual: $y_j \leq y_i + w_{ij}$. The objective function is also the maximization of $y_s - y_t$.
\\
\\
\textbf{\#2}
\\
\\
In a similar way of question \#1, let $w_e$ be the weight of every $e \in E$ and $x_e$ be $0-1$ variables such that $x_e = 1$ if the edge $e$ is in the matching and $x_e = 0$, otherwise. 
\begin{gather*}
x_e =
\begin{cases}
1 \tab\tab\text{ inclusion of edge } e \text{ in the matching} \\
0 \tab\tab\text{ otherwise}\\
\end{cases}
\end{gather*}
We need to choose at each step an augmenting path the produces the largest possible increase in total weight in $G$. Thus, the objective function maximizes the the weight of all edges $e$ in the matching and, because we have a path, we use constraints to limit one edge per vertex so that the path is created in the form $x_e \leq 1$, for all vertices $u$, such that $e=(u, v)$. The linear program is then:
$$
max \text{ } \sum_e w_e x_e
$$
$$
\sum_{e=(u,v)} x_e \leq 1 \text{ , } \forall u \in V
$$
\\
\\
\textbf{\#3}
\\
\textbf{\#4}
\\
\textbf{\#5}
\\
\end{document}