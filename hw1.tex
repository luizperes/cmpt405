\documentclass{article}
\PassOptionsToPackage{hyphens}{url}
\usepackage[hidelinks]{hyperref}
\usepackage{amsmath}
\usepackage{amsthm}
\usepackage{amssymb}
\usepackage{pgfplots}
\usepackage{algpseudocode}
\newcommand{\QED}{\hfill {\qed}}
\newcommand\tab[1][1cm]{\hspace*{#1}}
\usepackage{mathtools}
\DeclarePairedDelimiter\ceil{\lceil}{\rceil}
\DeclarePairedDelimiter\floor{\lfloor}{\rfloor}

\title{\#1 Assignment - CMPT 405}
\author{Luiz Fernando Peres de Oliveira - 301288301 - lperesde@sfu.ca}

\begin{document}

\maketitle

\textbf{\#1} - Let \textit{C} be the array containing all the possible coins \textit{\{1, 5, 10, 25, 100, 200 \}}. Let $V$ be the total change value.\\
\\
Algorithm:\\\\
\textbf{Input:} \textit{C, V}
\begin{algorithmic}
\State $d\gets$ sort C such that $d_1 \geq d_2 \geq ... \geq d_n$
\State $res \gets \emptyset;$  $i \gets 1$
\While{$V > 0$}
\If {$V \geq d_i$}
    \State $n_{coins}\gets   \floor*{\frac{V}{d_i}}$
    \State $V \gets V - (n_{coins} * d_i)$
    \State $res \gets res \cup \{ (d_i, n_{coins}) \}$
\EndIf
\State $i \gets i + 1$
\EndWhile
\State return $res$\\
\end{algorithmic}
\textit{Intuition:} $\forall (d_i, d_j) \in C$, $1 \leq i < j \leq n$, $d_i \geq 2*d_j$, meaning that if I the algorithm chooses any $d_j$ over any $d_i$, it will have to pick at least 2 times more coins for some value $V$ that satisfies both $d_i$ and $d_j$.\\
\begin{proof}
Now, imagine that the algorithm chooses $d_j$ over $d_i$, then we have two cases:\\
- Case 1. $d_i > V$:\\
\tab If so, we are done because there are no possible ways of choosing $d_i$ for value $V$.\\
- Case 2. $d_i \leq V$:\\
\tab If that was the case, we would have an optimal set $OPT$ such that $OPT_{i-1} \cup d_j \subseteq OPT$, which is not the case, once the iteration $i$ will happen before the iteration $j$, causing the algorithm to choose $d_i$ over $d_j$ (and never the opposite) for any value $V$ that satisfies both $d_i$ and $d_j$.\\\\
The only way to give more coins than the smallest possible number of coins for any change would be in a case where the algorithm chooses $d_j$ over a $d_i$, which will never happen. \qedhere
\end{proof}
\end{document}