\documentclass{article}
\PassOptionsToPackage{hyphens}{url}
\usepackage[hidelinks]{hyperref}
\usepackage{amsmath}
\usepackage{amsthm}
\usepackage{amssymb}
\usepackage{pgfplots}
\usepackage{algpseudocode}
\newcommand{\QED}{\hfill {\qed}}
\newcommand\tab[1][1cm]{\hspace*{#1}}
\usepackage{mathtools}
\DeclarePairedDelimiter\ceil{\lceil}{\rceil}
\DeclarePairedDelimiter\floor{\lfloor}{\rfloor}

\title{\#1 Assignment - CMPT 405}
\author{Luiz Fernando Peres de Oliveira - 301288301 - lperesde@sfu.ca}

\begin{document}

\maketitle

\textbf{\#1} - Let \textit{C} be the array containing all the possible coins \textit{\{1, 5, 10, 25, 100, 200 \}}. Let $V$ be the total change value.\\
\\
Algorithm:\\\\
\textbf{Input:} \textit{C, V}
\begin{algorithmic}
\State $d\gets$ sort C such that $d_1 \geq d_2 \geq ... \geq d_n$
\State $res \gets \emptyset;$  $i \gets 1$
\While{$V > 0$}
\If {$V \geq d_i$}
    \State $n_{coins}\gets   \floor*{\frac{V}{d_i}}$
    \State $V \gets V - (n_{coins} * d_i)$
    \State $res \gets res \cup \{ (d_i, n_{coins}) \}$
\EndIf
\State $i \gets i + 1$
\EndWhile
\State \textbf{return} $res$\\
\end{algorithmic}
\textit{Intuition:} $\forall (d_i, d_j) \in C$, $1 \leq i < j \leq n$, $d_i \geq 2*d_j$, meaning that if I the algorithm chooses any $d_j$ over any $d_i$, it will have to pick at least 2 times more coins for some value $V$ that satisfies both $d_i$ and $d_j$.\\
\begin{proof}
The only way to give more coins than the smallest possible number of coins for any change would be in a case where the algorithm chooses $d_j$ over a $d_i$ (see \textit{Intuition}). Now, imagine that the algorithm chooses $d_j$ over $d_i$, then we have two cases:\\
- Case 1. $d_i > V$:\\
\tab If so, we are done because there are no possible ways of choosing $d_i$ for value $V$.\\
- Case 2. $d_i \leq V$:\\
\tab If that was the case, we would have an optimal set $OPT$ such that $OPT_{i-1} \cup d_j \subseteq OPT$, which is not the case, once the iteration $i$ will happen before the iteration $j$, causing the algorithm to choose $d_i$ over $d_j$ (and never the opposite) for any value $V$ that satisfies both $d_i$ and $d_j$.\qedhere
\end{proof}
\textbf{\#2 a)}\\
Greedy approach to the fractional knapsack:\\
- $n$ objects and a knapsack\\
- item $i$ weighs $w_i > 0$ and has utility $u_i > 0$\\
- fill knapsack so as to \textbf{maximize} total utility/weight, not exceeding total capacity $W$\\
\\Algorithm approach:
\begin{algorithmic}
\State - sort items in decreasing order of their utility-to-weight ratio $u_i/w_i$
\State - repeatedly add item with max ratio $u_i/w_i$. If not possible to add the whole object, add a fraction $\alpha \in (0, 1)$ of it, if possible.
\end{algorithmic}
\begin{proof}
Let $K_{opt}$ be the optimal set of items in a knapsack and let $K_j$ be the chosen items after an iteration $j$, $0 \leq j \leq n$.\\\\
\textit{Base case:} $K_0$: $K_0$ is promising since the total number of chosen objects, in this case \textit{none}, does not exceed total capacity $W$, there exists some optimal $K_{opt}$ such that $K_0 \subseteq K_{opt} \subseteq K_0 \cup \{ i_1, i_2, ..., i_n\}$.\\
\\
\textit{Induction step}: Assume $K_{j-1}$. Since $K_{j-1}$ is promising for stage $j-1$, $K_{j-1} \subseteq K_{opt} \subseteq K_{j-1} \cup \{i_j, i_{j+1}, ..., i_n \}$. We want to show $K_j$. On a stage $j$ we have two cases:\\
\tab Case 1. $i_j$ is rejected. Then $K_{j-1} \cup \{ i_j \}$ or $K_{j-1} \cup \{ i_j * \alpha \}$  (any fraction $\alpha \in (0, 1)$ of $i_j$) exceed the capacity $W$; thus, $K_{j-1} = K_j$. Since $K_{j-1} \subseteq K_{opt}$ and $K_{opt}$ does not exceed the total capacity $W$, $i_j \notin K_{opt}$. So $K_j \subseteq K_{opt} \subseteq K_j \cup \{ i_{j+1}, i_{j+2}, ..., i_n\}$.\\
\tab Case 2. $i_j$  or $i_j * \alpha$ is added to $K_{j-1}$. Let item $i_{chosen}$ be $i_j$  or $i_j * \alpha$ (whichever was added to $K_{j-1}$). Then $K_{j-1} \cup \{i_{chosen}\}$ does not exceed the total capacity $W$ and we have $K_{j-1} \cup \{i_{chosen}\} = K_j$.\\
\tab Case 2.1. $i_{chosen} \in K_{opt}$. Then we have $K_j \subseteq K_{opt} \subseteq K_j \cup \{ i_{j+1}, i_{j+2}, ..., i_n\}$.
\tab Case 2.2. $i_{chosen} \notin K_{opt}$. We show that there is another maximum set of utility-to-weight items $K'_{opt}$ that witnesses the fact that $K_j$ is promising. For example, consider an item $i_{chosen}$ added to $K_{opt}$. This will exceed the capacity $W$ and the knapsack will contain at least one item of $\{i_{j+1}, i_{j+2}, ..., i_n\}$. \\
\tab \textit{Proof of claim:} $K_{opt}$ contains all elements of $K_{j-1}$ and can be obtained from $K_{j-1}$ by adding some items from the set $\{i_j, i_{j+1}, ..., i_n\}$. Adding $i_j$ does not exceed capacity $W$, so the excess in $K_{opt} \cup \{i_{chosen}\}$ must contain some elements other items in $\{i_{j+1}, i_{j+2}, ..., i_n\}$.\\\\
\end{proof}
\textbf{\#2 b)}\\
\begin{tabular}{|c|c|c| }
 \hline
 item & utility & weight \\
 \hline
 1 & 2 & 1 \\
 \hline
 2 & 1000 & 1000\\
 \hline
\end{tabular}
$W = 1000$
\\$K_{opt} = \{i_2\} ($utility $= 1000)$, $K_{greedy} = \{i_1\} ($utility $= 2)$\\
\textbf{\#3}\\
The idea here is to use the greedy approach of the \textbf{set cover} problem. Let $U$ be the the collection with all tiles $a_{ij}$ and let two tiles $a_{ij}, a_{i+1k}$ be adjancent if it is possible to color them horizontally, meeting the algorithm criteria.\\\\
Algorithm:\\\\
\textbf{Input:} \textit{U}
\begin{algorithmic}
\For{$i$ \textbf{from} $1$ \textbf{to} $K$}
  \For{$j$ \textbf{from} $1$ \textbf{to} $n_i$}
    \State $arr[i] \gets$ sort tiles such that $a_{ij} \geq a_{ij+1} \geq ... \geq a_{i_ni}$
  \EndFor
\EndFor
\State $S \gets$ create sets $S_1, S_2,...,S_{n_{overlaps}}$ for each possible row using adjacent tiles in $arr$.
\State $C \gets 0$
\While{all tiles are not covered}
  \State choose $s \in S$ such that $s$ contains most uncovered tiles.
  \State mark the tiles in $s$ as covered
  \State $C \gets C + 1$
\EndWhile
\State \textbf{return} C\\\\
\end{algorithmic}
\textit{Counterexample:}\\
\\\textit{Columns:}\\
$col$ $1 : a_{11} = 0.325$, $a_{12} = 0.225$, $a_{13} = 0.225$, $a_{14} = 0.225$\\ 
$col$ $2 : a_{21} = 0.45$, $a_{22} = 0.225$, $a_{23} = 0.225$, $a_{24} = 0.1$\\
$col$ $3 : a_{31} = 0.225$, $a_{32} = 0.225$, $a_{33} = 0.225$, $a_{34} = 0.225$, $a_{35} = 0.1$\\
$col$ $4 : a_{41} = 0.325$, $a_{42} = 0.225$, $a_{43} = 0.1125$, $a_{44} = 0.1125$, $a_{45} = 0.1125$, $a_{46} = 0.1125$\\
$col$ $5 : a_{51} = 0.55$, $a_{52} = 0.45$\\
$col$ $6 : a_{61} = 0.33333,$ $a_{62} = 0.33333$, $a_{63} = 0.33333$\\
$col$ $7 : a_{71} = 0.44444$, $a_{72} = 0.22222$, $a_{73} = 0.22222$, $a_{74} = 0.11112$\\
\\\textit{Created Sets:}\\
$S_1 = \{a_{11}$, $a_{21}$, $a_{31}$, $a_{41}$, $a_{51}$, $a_{61}$, $a_{71} \}$\\
$S_2 = \{a_{11}$, $a_{21}$, $a_{32}$, $a_{41}$, $a_{51}$, $a_{61}$, $a_{71} \}$\\
$S_3 = \{a_{12}$, $a_{21}$, $a_{32}$, $a_{42}$, $a_{51}$, $a_{62}$, $a_{71} \}$\\
$S_4 = \{a_{12}$, $a_{22}$, $a_{33}$, $a_{42}$, $a_{51}$, $a_{62}$, $a_{72} \}$\\
$S_5 = \{a_{13}$, $a_{22}$, $a_{33}$, $a_{43}$, $a_{52}$, $a_{62}$, $a_{72}\}$\\
$S_6 = \{a_{13}$, $a_{23}$, $a_{34}$, $a_{44}$, $a_{52}$, $a_{63}$, $a_{73}\}$\\
$S_7 = \{a_{14}$, $a_{23}$, $a_{34}$, $a_{45}$, $a_{52}$, $a_{63}$, $a_{73}\}$\\
$S_8 = \{a_{14}$, $a_{24}$, $a_{35}$, $a_{46}$, $a_{52}$, $a_{63}$, $a_{74}\}$\\
\end{document}