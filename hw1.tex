\documentclass{article}
\PassOptionsToPackage{hyphens}{url}
\usepackage[hidelinks]{hyperref}
\usepackage{amsmath}
\usepackage{amsthm}
\usepackage{amssymb}
\usepackage{pgfplots}
\usepackage{algpseudocode}
\newcommand{\QED}{\hfill {\qed}}
\newcommand\tab[1][1cm]{\hspace*{#1}}
\usepackage{mathtools}
\DeclarePairedDelimiter\ceil{\lceil}{\rceil}
\DeclarePairedDelimiter\floor{\lfloor}{\rfloor}

\title{\#1 Assignment - CMPT 405}
\author{Luiz Fernando Peres de Oliveira - 301288301 - lperesde@sfu.ca}

\begin{document}

\maketitle

\textbf{\#1} - Let \textit{C} be the array containing all the possible coins \textit{\{1, 5, 10, 25, 100, 200 \}}. Let $V$ be the total change value.\\
\\
Algorithm:\\
\textbf{Input:} \textit{C, V}
\begin{algorithmic}
\State $d\gets$ sort C such that $d_1 \geq d_2 \geq ... \geq d_n$
\State $res \gets \emptyset;$  $i \gets 1$
\While{$V > 0$}
\If {$V \geq d_i$}
    \State $n_{coins}\gets   \floor*{\frac{V}{d_i}}$
    \State $V \gets V - (n_{coins} * d_i)$
    \State $res \gets res \cup \{ (d_i, n_{coins}) \}$
\EndIf
\State $i \gets i + 1$
\EndWhile
\State \textbf{return} $res$\\
\end{algorithmic}
\textit{Intuition:} For all $d_i, d_j \in C$, $1 \leq i < j \leq n$, $d_i \geq 2*d_j$, meaning that if I the algorithm chooses any $d_j$ over any $d_i$, it will have to pick at least 2 times more coins for some value $V$ that satisfies both $d_i$ and $d_j$.\\
\begin{proof}
The only way to give more coins than the smallest possible number of coins for any change would be in a case where the algorithm chooses $d_j$ over a $d_i$ (see \textit{Intuition}). Now, imagine that the algorithm chooses $d_j$ over $d_i$, then we have two cases:\\
- Case 1. $d_i > V$:\\
\tab If so, we are done because there are no possible ways of choosing $d_i$ for value $V$.\\
- Case 2. $d_i \leq V$:\\
\tab If that was the case (as a mean of contradiction), we would have an optimal set $OPT$ such that $OPT_{i-1} \cup d_j \subseteq OPT$, which is not the case, once the iteration $i$ will happen before the iteration $j$, causing the algorithm to choose $d_i$ over $d_j$ (and never the opposite) for any value $V$ that satisfies both $d_i$ and $d_j$.\qedhere
\end{proof}
\textbf{\#2 a)}\\
Greedy approach to the fractional knapsack:\\
- $n$ objects and a knapsack\\
- item $i$ weighs $w_i > 0$ and has utility $u_i > 0$\\
- fill knapsack so as to \textbf{maximize} total utility/weight, not exceeding total capacity $W$\\
\\Algorithm approach:
\begin{algorithmic}
\State - sort items in decreasing order of their utility-to-weight ratio $u_i/w_i$
\State - repeatedly add item with max ratio $u_i/w_i$. If not possible to add the whole object, add a fraction $\alpha \in (0, 1)$ of it, if possible.
\end{algorithmic}
\begin{proof}
Let $K_{opt} \subseteq \{i_1, i_2, i_3,..., i_n\}$ be the optimal set of items in a knapsack and let $K_j$ be the chosen items after an iteration $j$, $0 \leq j \leq n$. Let $K_j$ be considered "promising" if $K_j \subseteq K_{opt}$.\\\\
\textit{Base case:} $K_0$: $K_0$ is promising since the total number of chosen objects, in this case \textit{none}, does not exceed total capacity $W$. Thus, there exists some optimal $K_{opt}$ such that $K_0 \subseteq K_{opt} \subseteq K_0 \cup \{ i_1, i_2, ..., i_n\}$.\\
\\
\textit{Induction step}: Assume $K_{j-1}$ is promising for stage $j-1$, meaning that $K_{j-1} \subseteq K_{opt} \subseteq K_{j-1} \cup \{i_j, i_{j+1}, ..., i_n \}$. We want to show $K_j$. On a stage $j$ we have two cases:\\
\tab Case 1. $i_j$ is rejected. Then $K_{j-1} \cup \{ i_j \}$ or $K_{j-1} \cup \{ i_j * \alpha \}$  (any fraction $\alpha \in (0, 1)$ of $i_j$) exceed the capacity $W$; thus, $K_{j-1} = K_j$. Since $K_{j-1} \subseteq K_{opt}$ and $K_{opt}$ does not exceed the total capacity $W$, $i_j \notin K_{opt}$. So $K_j \subseteq K_{opt} \subseteq K_j \cup \{ i_{j+1}, i_{j+2}, ..., i_n\}$.\\
\tab Case 2. $i_j$  or $i_j * \alpha$ is added to $K_{j-1}$. Let item $i_{chosen}$ be $i_j$  or $i_j * \alpha$ (whichever was added to $K_{j-1}$). Then $K_{j-1} \cup \{i_{chosen}\}$ does not exceed the total capacity $W$ and we have $K_{j-1} \cup \{i_{chosen}\} = K_j$.\\
\tab Case 2.1. $i_{chosen} \in K_{opt}$. Then we have $K_j \subseteq K_{opt} \subseteq K_j \cup \{ i_{j+1}, i_{j+2}, ..., i_n\}$.
\tab Case 2.2. $i_{chosen} \notin K_{opt}$. We show that there is another maximum set of utility-to-weight items $K'_{opt}$ that witnesses the fact that $K_j$ is promising. For example, consider an item $i_{chosen}$ added to $K_{opt}$. This will exceed the capacity $W$ and the knapsack will contain at least one item of $\{i_{j+1}, i_{j+2}, ..., i_n\}$. \\
\tab \textit{Proof of claim (Case $2.2$):} $K_{opt}$ contains all elements of $K_{j-1}$ and can be obtained from $K_{j-1}$ by adding some items from the set $\{i_j, i_{j+1}, ..., i_n\}$. Adding $i_j$ does not exceed capacity $W$, so the excess in $K_{opt} \cup \{i_{chosen}\}$ must contain some elements other items in $\{i_{j+1}, i_{j+2}, ..., i_n\}$.\\
\end{proof}
\textbf{\#2 b)}\\
\begin{tabular}{|c|c|c| }
 \hline
 item & utility & weight \\
 \hline
 1 & 2 & 1 \\
 \hline
 2 & 1000 & 1000\\
 \hline
\end{tabular}
$W = 1000$
\\$K_{opt} = \{i_2\} ($utility $= 1000)$, $K_{greedy} = \{i_1\} ($utility $= 2)$\\
\textbf{\#3}\\
The idea here is to use the greedy approach to check ahead and count the number of tiles $a_{xy}$, $1 \leq i < x \leq k$, $1 \leq y \leq n_x$ adjacent to a tile $a_{ij}$.  Basically, if a tile $a_{ij}$ touches the bounds of a tile $a_{xy}$, we say they are adjacent. The $MAX$ number of adjacent tiles is then then minimum required number of colors for our solution. For example, if we have the tiles $a_{11} = 0.6$, $a_{12} = 0.4$ and $a_{21} = 0.35$, $a_{22} = 0.35$, $a_{23} = 0.3$, then $a_{11}$ is adjacent to $a_{21}$ and $a_{22}$ and $a_{12}$ is adjacent to $a_{22}$ and $a_{23}$. The minimum number of colours required would then be $4$ in this case.\\\\
Algorithm:\\\\
\textbf{Input:} \textit{wall, k}
\begin{algorithmic}
\For{$i$ \textbf{from} $1$ \textbf{to} $k$}
  \For{$j$ \textbf{from} $1$ \textbf{to} $n_i$}
    \State $wall_i \gets$ sort tiles such that $a_{ij} \geq a_{ij+1} \geq ... \geq a_{in_i}$
  \EndFor
\EndFor
\State $C \gets 0$
\For{$i$ \textbf{from} $1$ \textbf{to} $k - 1$}
  \State $c \gets 0$
  \For{$j$ \textbf{from} $1$ \textbf{to} $n_i$}
    \State $x \gets i+1$
    \For{$y$ \textbf{from} $1$ \textbf{to} $n_x$}
      \If{$a_{ij}$ adjacent to $a_{xy}$}
        \State $c \gets c+1$
      \Else
        \State \textbf{break} \textit{ //not adjacent to $a_{xy+1}$ through $a_{xn_{x}}$ as well}
      \EndIf
    \EndFor
  \EndFor
  \State $C \gets MAX(C, c)$
\EndFor
\State \textbf{return} $C$\\
\end{algorithmic}
\textit{Counterexample:}\\
\\\textit{Suppose:}\\
$col$ $1 : a_{11} = 0.6$, $a_{12} = 0.4$\\
$col$ $2 : a_{21} = 0.4$, $a_{22} = 0.4$, $a_{23} = 0.2$\\
\\
Our Greedy algorithm selects $a_{11}$ as being adjacent to $a_{21}$ and $a_{22}$. After that, it selects $a_{12}$ as being adjacent to $a_{22}$ and $a_{23}$, returning $4$ as the minimal number of colours. The algorithm could not rearrange $a_{22}$ and $a_{23}$ so that $a_{11}$ would be adjacent to $a_{21}$ and $a_{23}$; and $a_{12}$ would be adjacent to $a_{22}$, returning $3$, the optimal number of colors for this problem. Thus, the Greedy approach given is not optimal. See below:\\
\\
$W_{greedy} = \{(a_{11}, a_{21}), (a_{11}, a_{22}), (a_{12}, a_{22}), (a_{12}, a_{23})\}$. \textbf{min:} $4$ colors\\
$W_{opt} = \{(a_{11}, a_{21}), (a_{11}, a_{23}), (a_{12}, a_{22})\}$. \textbf{min:} 3 colors\\
\\
\textbf{\#4}\\
\textit{Definition:} For $0 \leq i \leq m$, $0 \leq j \leq n$, define $M[i, j]$ as the optimal number of paths in the Cartesian plane from $(0, 0)$ to $(i, j)$ that uses the combined number of steps of type $U$(up, $M[i-1, j]$), $R$(right, $M[i, j-1]$) and $D$(diagonal, $M[i-1, j-1]$). The optimal number of paths from $(0,0)$ to $(m, n)$ is then $M[m, n]$.\\
\\
\textit{Recurrence:}
\begin{gather*}
M[i, j] =
\begin{cases}
1 \tab\tab\tab\tab\tab\tab\tab $if $i = 0$ or $j = 0 \\
M[i-1, j] + M[i, j-1] + M[i-1, j-1] \tab $ otherwise $
\end{cases}
\end{gather*}\\
Algorithm:\\
\textbf{Input:} \textit{m, n}
\begin{algorithmic}
\State Make matrix $M$ with dimensions $m \times n$
\For{$i$ \textbf{from} $0$ \textbf{to} $m$}
  \State $M[i, 0] \gets 1$
\EndFor
\For{$j$ \textbf{from} $0$ \textbf{to} $n$}
  \State $M[0, j] \gets 1$
\EndFor
\For{$i$ \textbf{from} $1$ \textbf{to} $m$}
  \For{$j$ \textbf{from} $1$ \textbf{to} $n$}
    \State $M[i, j] \gets M[i-1, j] + M[i, j-1] + M[i-1, j-1]$
  \EndFor
\EndFor
\State \textbf{return} $M[m, n]$\\
\end{algorithmic}
\textit{Running time:} The running time of the loops are respectively $O(m)$, $O(n)$ and $O(mn)$. All operations on $M$ (inside the loops) are constant ($O(1)$) and therefore the total running time of the algorithm is $O(mn)$.\\\\
\textit{Expression}:\\
For $d$ steps of type $D$(diagonal), there must have $m - d$ steps of type $U$ and $n - d$ steps of type $R$, in order to reach $(m, n)$. It can then be represented by:\\
$$M(m, n) = \sum_{d=0}^{min(m, n)} {m + n - d \choose m} {m \choose d} $$\\\\
\textbf{\#5}\\
The idea of the problem is to use the Greedy approach on the \textbf{set cover} problem to get maximal rectangles, meaning that if I have a position $p_{v_i} = 0$, where $v$ is a leaf of $T$ and $i$ is an index of $u$'s binary string, and I also have $p_{u_i} = 0$, for a leaf $u$ sibling of $v$, then I can maximize a rectangle for a $w$ parent of $v$ and $u$, such that $p_{w_i} = 0$ (the same holds for sequences of $0$'s). We start the algorithm by setting up the sets for the set cover problem such that the longest number of consecutive zeroes and ones stay make a group within a string (e.g $S_v$ for $v = 1011000$ would be $S_v = \{1, 0, 11, 000\}$. We then traverse $T$ such that we get maximal rectangles, see the algorithm below: \\\\
Algorithm:\\
\textbf{Input:} \textit{T}
\begin{algorithmic}
\For{\textbf{each} $v \in T, children(v) = \emptyset $}
  \State $S_v \gets$ group $0$'s and $1$'s $\in v$ as a list, such that no $0$ in position $i$ is followed by another (group of) $0$ in position $i+1$ and no $1$ in position $i$ is followed by another (group of) $1$ in position $i+1$
\EndFor
\For{\textbf{each} $w \in T_{postorder}, children(w) = \{u, v\}$}
  \State $S_w \gets$ matching-indexes $0$'s in both $S_v$ and $S_u$
  \State $S_v \gets S_v - S_w$
  \State $S_u \gets S_u - S_w$ 
\EndFor
\State $Res \gets \emptyset$
\For{\textbf{each} $v \in T$}
  \If{$S_v \neq \emptyset$}
    \State $rectangle \gets \forall(v, (i, j)) \in S_v$ where $p_{v_ij} = \{0\}^{|v|}$
    \State $Res \gets Res$ $\cup$ $rectangle$
  \EndIf
\EndFor
\State \textbf{return} $Res$\\
\end{algorithmic}
\textit{Running time:}\\
Let $b$ be the number of bits in a binary string on the leaves of $T$. The algorithm then takes $O(nlogn)$ to traverse $T$ and $O(b)$ for each set-related operation inside the loops. Therefore, the total running time of the algorithm is $O(bnlogn)$.
\end{document}