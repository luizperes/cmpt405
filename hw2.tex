\documentclass{article}
\PassOptionsToPackage{hyphens}{url}
\usepackage[hidelinks]{hyperref}
\usepackage{amsmath}
\usepackage{amsthm}
\usepackage{amssymb}
\usepackage{pgfplots}
\usepackage{algpseudocode}
\newcommand{\QED}{\hfill {\qed}}
\newcommand\tab[1][1cm]{\hspace*{#1}}
\usepackage{mathtools}
\DeclarePairedDelimiter\ceil{\lceil}{\rceil}
\DeclarePairedDelimiter\floor{\lfloor}{\rfloor}

\title{\#2 Assignment - CMPT 405}
\author{Luiz Fernando Peres de Oliveira - 301288301 - lperesde@sfu.ca}

\begin{document}

\maketitle
\textbf{\#1}
\\
First, we draw the table with cost $c$ of multiplying two matrices in the dimensions $\{ 1 \times 1, 1 \times d, d \times 1, d \times d\}$
\\
\begin{center}
\begin{tabular}{ |c|c|c|c| } 
 \hline
 $m_1$ & $m_2$ & $m_{res}$ & cost \\
 \hline
 $1 \times 1$ & $1 \times 1$ & $1 \times 1$ & $1$ \\
 \hline
 $1 \times 1$ & $1 \times d$ & $1 \times d$ & $d$ \\
 \hline
 $1 \times d$ & $d \times 1$ & $1 \times 1$ & $d$ \\
 \hline
 $1 \times d$ & $d \times d$ & $1 \times d$ & $d^2$ \\
 \hline
 $d \times 1$ & $1 \times 1$ & $d \times 1$ & $d$ \\
 \hline
 $d \times 1$ & $1 \times d$ & $d \times d$ & $d^2$ \\
 \hline
 $d \times d$ & $d \times 1$ & $d \times 1$ & $d^2$ \\
 \hline
 $d \times d$ & $d \times d$ & $d \times d$ & $d^3$ \\
 \hline
\end{tabular}
\end{center}
\textit{Intuition:}
\\
The algorithm should always choose the 
\textbf{\#2}
\\
\\
\textbf{\#3}
\\
\\
\textbf{\#4}\\
\textit{Definition:} Let $A$ be an array with size $n+1$ and $s$ be a sequence of integers. Initialize $A[0] = - \infty$ and for $1 \leq i \leq n$, define $A[i]$ as the largest contiguous subsequence sum in $s$ after an iteration $i$.  At the end, the largest possible sum will then be the highest element in $A$.\\
\\
\textit{Recurrence:}
\begin{gather*}
A[i] =
\begin{cases}
-\infty \tab\tab\tab\tab\text{   } $if $i = 0 \\
\text{max } \{A[i-1] + s[i], s[i]\} \tab $ otherwise $
\end{cases}
\end{gather*}\\
Algorithm:\\
\textbf{Input:} \textit{s, n}
\begin{algorithmic}
\State Make array $A$ of size $n + 1$
\State $A[0] \gets -\infty$
\State insert $none$ in the index $0$ of $s$ \textit{// make $|s| = |A|$ for the loop}
\State $best_i \gets 0$
\For{$i$ \textbf{from} $1$ \textbf{to} $n$}
  \State $A[i] \gets \text{max } \{A[i-1]+s[i], s[i] \}$
  \If{$A[i] > A[i-1]$}
    \State $best_i \gets i$
  \EndIf
\EndFor
\State $s' \gets \emptyset$ \textit{ // find best subsequence index set $s'$}
\While{$A[best_i] = A[best_i - 1] + s[best_i]$}
  \State $s' \gets s' \cup \{ best_i \}$
  \State $best_i \gets best_i - 1$
\EndWhile
\State $s' \gets s' \cup best_i$ \textit{// add the lowest index of subsequence}
\State \textbf{return} $s'$\\
\end{algorithmic}
\textit{Demonstration:}
\\\\
$s = $
\begin{tabular}{ |c|c|c|c|c|c|c| } 
 \hline
 none & -2 & 11 & -4 & 13 & -5 & -2 \\
 \hline
\end{tabular}
\\
$A = $
\begin{tabular}{ |c|c|c|c|c|c|c| } 
 \hline
 $-\infty$ & -2 & 11 & 7 & \textbf{20} & 15 & 13 \\
 \hline
\end{tabular}
\\
$s' = \{ 2, 3, 4\}$
\\
\\
At the end, as $s'$ demonstrates, we will have the range $2..4$ comprising the largest possible sum of a contiguous subsequence in $s$.
\\
\\
\textit{Running time:} The running time of the loops are $O(n)$. All operations on $A$ (inside the loops) are constant ($O(1)$) and therefore the total running time of the algorithm is $O(n)$.\\\\
\textbf{\#5}
\\
\\
\end{document}