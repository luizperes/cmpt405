\documentclass{article}
\PassOptionsToPackage{hyphens}{url}
\usepackage[hidelinks]{hyperref}
\usepackage{amsmath}
\usepackage{amsthm}
\usepackage{amssymb}
\usepackage{pgfplots}
\usepackage{algpseudocode}
\newcommand{\QED}{\hfill {\qed}}
\newcommand\tab[1][1cm]{\hspace*{#1}}
\usepackage{mathtools}
\DeclarePairedDelimiter\ceil{\lceil}{\rceil}
\DeclarePairedDelimiter\floor{\lfloor}{\rfloor}

\title{\#2 Assignment - CMPT 405}
\author{Luiz Fernando Peres de Oliveira - 301288301 - lperesde@sfu.ca}

\begin{document}

\maketitle
\textbf{\#1}
\\
First, we draw the table with cost $c$ of multiplying two matrices in the dimensions $\{ 1 \times 1, 1 \times d, d \times 1, d \times d\}$
\\
\begin{center}
\begin{tabular}{ |c|c|c|c| } 
 \hline
 $m_1$ & $m_2$ & $m_{res}$ & cost \\
 \hline
 $1 \times 1$ & $1 \times 1$ & $1 \times 1$ & $1$ \\
 \hline
 $1 \times 1$ & $1 \times d$ & $1 \times d$ & $d$ \\
 \hline
 $1 \times d$ & $d \times 1$ & $1 \times 1$ & $d$ \\
 \hline
 $1 \times d$ & $d \times d$ & $1 \times d$ & $d^2$ \\
 \hline
 $d \times 1$ & $1 \times 1$ & $d \times 1$ & $d$ \\
 \hline
 $d \times 1$ & $1 \times d$ & $d \times d$ & $d^2$ \\
 \hline
 $d \times d$ & $d \times 1$ & $d \times 1$ & $d^2$ \\
 \hline
 $d \times d$ & $d \times d$ & $d \times d$ & $d^3$ \\
 \hline
\end{tabular}
\end{center}
\textit{Intuition:}
\\
The algorithm should always choose the lowest possible cost $c \in \{1, d, d^2, d^3\}$ because according to the table above, for any two given fixed matrices $m_1$, $m_2$ in a formula $\varphi$, where $\varphi$ is the formula multiplying $n$ matrices with dimensions $\{ 1 \times 1, 1 \times d, d \times 1, d \times d\}$, if it chooses the lowest cost operation, it will also generate a matrix $m_{res}$ that, by multiplying another (possible) matrix $m_3$, will have cost $c' \geq c$. If $c' = c$, we keep the same cost, which is good, since we picked the lowest cost $c$ at first. If $c' > c$, then, by the table above, $c' = cd$, which is also the lowest possible cost. As a mean of contradiction, if the algorithm chooses a matrix for which $c' > cd$, then the matrix would have non-compatible dimensions. Therefore, it is important that formula $\varphi$ be a valid formula. Because we know that $\varphi$ is always valid, we know that we can group (or parenthesize) the matrices by their dimensions and you will see that it can be done in linear time. The approach of the algorithm is the that with a stack $S$, every time it comes across a matrix $m_i = (1 \times x), x \in \{ 1, d\}, 1 \leq i \leq n$, it pushes $i$ onto $S$, as the idea of the algorithm is to prioritize low costs. At the end, it subdivides $\varphi$ in ascending order subformulas $\varphi' = \varphi_1', ... \varphi_m'$ (recall that we push onto $S$ the indexes of $(1 \times x)$ matrices). For each $\varphi_j' \in \varphi'$, we left-associate the matrices inside them. See the example and algorithm below:
\newpage
$\varphi = (d \times 1)(1 \times 1)(1 \times d)(d \times 1)(1 \times 1)$
\\
$S = $
\begin{tabular}{ |c|c|c| } 
 \hline
 1 & 2 & 4\\
 \hline
\end{tabular}
\\
$\varphi' = $
\begin{tabular}{ |c|c|c|c|c } 
 \hline
 $(d \times 1)$ & $(1 \times 1)$ & $(1 \times d)(d \times 1)$ & $(1 \times 1)$ \\
 \hline
\end{tabular}
\\
$res = (d \times 1) \bigg((1 \times 1) \bigg( \bigg( (1 \times d) (d \times 1) \bigg) (1 \times 1) \bigg) \bigg) $
\\
Algorithm:\\
\textbf{Input:} $\varphi, n$
\begin{algorithmic}
\State Make stack $S$
\For{$i$ from $1 $ to $n$}
  \State $(row, col) \gets \text{ matrix}_i$
  \If{row = 1}
    \State push $i$ onto $S$
  \EndIf
\EndFor
\State $\varphi' = \varphi$
\While{$S \neq \emptyset$}
  \State $t \gets \text{ pop } S$
  \State $v \gets \text{ top } S$ \textit{// check top of $S$ without popping}
  \State $\text{diff} \gets t - v$
  \If{$\text{diff} > 0$}
    \State $\varphi' \gets$ break $\varphi'$ and add parenthesis at pos $v$
    \State $\text{diff} \gets \text{diff} - 1$
    \While{$\text{diff} > 0$}
      \State $\varphi' \gets$ break $\varphi'$ and add parenthesis by left-associating extra \tab\tab\ matrices 
      \State $\text{diff} \gets \text{diff} - 1$
    \EndWhile
  \EndIf
\EndWhile
\State \textbf{return} $\varphi'$
\\
\end{algorithmic}
\textbf{\#2}
\\
\textit{Definition:} Let $A$ be an array with size $n+1$. Initialize $A[0] = 1$, $A[1] = 1$. Define $A[i]$ as the number of possible orders in which you can multiply after an iteration $i$.  At the end, the number total of possible orders will be $A[n]$.
\\
\\
\textit{Recurrence:}
\\
\begin{gather*}
A[i] =
\begin{cases}
1 \tab\tab\tab\tab\text{   } $if $ n = 1 $ or $ n = 2 \\
\sum_{j=1}^{i} A[j-1]*A[i-j] \tab n \geq 2
\end{cases}
\end{gather*}
\newpage
Algorithm:\\
\textbf{Input:} \textit{n}
\begin{algorithmic}
\State Make array $A$ of size $n + 1$
\State $A[0] \gets 1$
\State $A[1] \gets 1$
\For{$i $ \textbf{ from } $2$ \textbf{ to } $n$}
  \For{$j $ \textbf{ from } $1$ \textbf{ to } $i$}
    \State $A[i] \gets A[j-1]*A[i-j]$
  \EndFor
\EndFor
\State \textbf{return} $A[n]$
\end{algorithmic}
\textit{Closed form:}
\\
The closed form for the catalan number is (as per [2]):\\
$C_n = \frac{1}{n+1}\bigg(\frac{2n}{n}\bigg)$
\\
\\
\textbf{\#3}\\
\textit{Definition:} Let $A$ be an array with size $n+1$ and $s$ be a sequence of integers. Initialize $A[0] = - \infty$ and for $1 \leq i \leq n$, define $A[i]$ as the largest contiguous subsequence sum in $s$ after an iteration $i$.  At the end, the largest possible sum will then be the highest element in $A$.\\
\\
\textit{Recurrence:}
\begin{gather*}
A[i] =
\begin{cases}
-\infty \tab\tab\tab\tab\text{   } $if $i = 0 \\
\text{max } \{A[i-1] + s[i], s[i]\} \tab $ otherwise $
\end{cases}
\end{gather*}\\
Algorithm:\\
\textbf{Input:} \textit{s, n}
\begin{algorithmic}
\State Make array $A$ of size $n + 1$
\State $A[0] \gets -\infty$
\State insert $none$ in the index $0$ of $s$ \textit{// make $|s| = |A|$ for the loop}
\State $best_i \gets 0$
\For{$i$ \textbf{from} $1$ \textbf{to} $n$}
  \State $A[i] \gets \text{max } \{A[i-1]+s[i], s[i] \}$
  \If{$A[i] > A[i-1]$}
    \State $best_i \gets i$
  \EndIf
\EndFor
\State $s' \gets \emptyset$ \textit{ // find best subsequence index set $s'$}
\While{$A[best_i] = A[best_i - 1] + s[best_i]$}
  \State $s' \gets s' \cup \{ best_i \}$
  \State $best_i \gets best_i - 1$
\EndWhile
\State $s' \gets s' \cup best_i$ \textit{// add the lowest index of subsequence}
\State \textbf{return} $s'$\\
\end{algorithmic}
\textit{Demonstration:}
\\\\
$s = $
\begin{tabular}{ |c|c|c|c|c|c|c| } 
 \hline
 none & -2 & 11 & -4 & 13 & -5 & -2 \\
 \hline
\end{tabular}
\\
$A = $
\begin{tabular}{ |c|c|c|c|c|c|c| } 
 \hline
 $-\infty$ & -2 & 11 & 7 & \textbf{20} & 15 & 13 \\
 \hline
\end{tabular}
\\
$s' = \{ 2, 3, 4\}$
\\
\\
At the end, as $s'$ demonstrates, we will have the range $2..4$ comprising the largest possible sum of a contiguous subsequence in $s$.
\\
\\
\textit{Running time:} The running time of the loops are $O(n)$. All operations on $A$ (inside the loops) are constant ($O(1)$) and therefore the total running time of the algorithm is $O(n)$.
\\
\\
\textbf{\#4}
\\
The idea of the problem is to take any node on the tree $T$, consider it as a root and then do a \textbf{postorder traversal} applying a recurrence similar do the one on question $\#3$. Consider all vertices are labelled $v_1, ..., v_i, ..., v_n$, $1 \leq i \leq n$. Let $A$ be an array that keeps the highest sum of a subtree with root on vertex $v$. As we are doing a postorder traversal, when we compute $A[v_p]$,  $v_p$ being a parent vertex, we will have the sum of the children already computed. We only include $A[v_c]$ on the sum of the children of $v_p$ if $A[v_c] > 0$, and, if this holds true, $v_c$ is also denominated $v_c{+}$, for all children $v_c$ of $v_p$. The recurrence and algorithm would then be:
\\
\begin{gather*}
A[v_i] =
\begin{cases}
\text{max } \{ \text{ } \sum_{v_c^{+} \in \textit{children} (v_i)} A[v_c^{+}] + w(v_i) , w(v_i) \text{ } \}
\end{cases}
\end{gather*}\\
\\
Algorithm:\\
\textbf{Input:} \textit{T}
\begin{algorithmic}
\State Make array $A$ of size $n$
\State $best_i \gets 0$ \textit{// $v_{best_i}$ will be the root of $H$}
\For{\textbf{each} $v_i \in T_{postorder}$}
  \State $A[v_i] \gets \text{max } \{ \text{ } \sum_{c^{+} \in \textit{children} (v_i)} A[v_c^{+}] + w(v_i) , w(v_i) \text{ } \}$
  \If{$A[v_i] > A[best_i]$}
    \State $best_i \gets i$
  \EndIf
\EndFor
\State $H \gets$ Recursively do $(v_i, $ all $v_c^{+} \in $ children$(v_i))\}$ starting on root $A[v_{best_i}]$
\State \textbf{return} $H, A[v_{best_i}]$\\
\end{algorithmic}
\textbf{\#5}
\\
\textit{Intuition:}
\\
The intuition here is that if there exists any opportunities to start with an initial capital of $1$ unit of $i$ and exchange it until we get more than $1$, then there is a cycle among some edges in the graph $G$ such that we end up with more units than we started. For some edge $e_{ij}$, we will then have a path that looks like: $e_{ij} \rightarrow e_{jk} \rightarrow e_{kl} \rightarrow e_{li}$ such that cost $c = e_{ij} * e_{jk} * e_{kl} * e_{li} > 1$.  This path is a cycle because it starts and finishes in $i$ and that is the reason why we can exchange it back to $i$. If there is no cycle, then there is no possible ways to get back to currency $i$. A good approach for this algorithm is to pre-process our edges $e_{ij}$ (as per $[1])$ and then run Bellman-Ford algorithm to find whether or not there are any negative cycles. The pre-processing starts by using logarithms (it holds true for the problem because the logarithm function is a monotonic increasing function) to convert the multiplication into a summation and then setting the new values so we can run Bellman-Ford: 
\begin{equation}
\prod_{1 \leq i < j \leq k} e_{ij}  \tab\text{ to }\tab \sum_{1 \leq i < j \leq k} ln(e_{ij}) 
\end{equation}
\\
Because we want to find negative cycles, we will take equation $(1)$ and invert the sign, because the log is negative for the range $(0, 1)$ and we want the opposite: we want to find negative cycles if the total amount is greater than 1. So we have :
\begin{equation}
\sum_{1 \leq i < j \leq k} - ln(e_{ij}) 
\end{equation}
\\
After the pre-processing, it is possible to run Bellman-Ford to find if there is a negative cycle. If so, we know that there are opportunities for currency $i$ to be exchanged and get a value greater than the initial one. We repeat that for all currencies. See below:
\\
Algorithm:\\
\textbf{Input:} \textit{G}
\begin{algorithmic}
\State $s \gets \emptyset$
\For{\textbf{each} node $i \in G$}
  \State nodes $\gets$ all relaxed nodes on $n_{th}$ iterarion of Bellman-Ford (source on $i$)
  \State cycle$^{-}$ $\gets$ \textbf{false}
  \While{nodes $\neq \emptyset$ \textbf{and} \textbf{not} cycle$^{-}$}
    \State $j \gets $ get any element from nodes
    \State nodes $\gets$ nodes$- \{j\}$
    \State $T' \gets $ run BFS with source $j$
    \If{$i$ is reachable from $j$ (using $T')$}
      \State cycle$^{-}$ $\gets$ \textbf{true}
    \EndIf
  \EndWhile
  \If{cycle$^{-}$}
    \State $s \gets s \cup {i}$
  \EndIf
\EndFor
\State \textbf{return} $s$
\end{algorithmic}
\textbf{References}
\\
\\ $[1]$ - \textit{Coursera: Algorithms on Graphs (https://www.coursera.org/lecture/algorithms-on-graphs/infinite-arbitrage-MrQ2H)}
\\ $[2]$ - \textit{Catalan number - https://en.wikipedia.org/wiki/Catalan\_number}
\end{document}